%\todo{RONALD and ALETHIA}

%{\it including assessment against functional requirements - elicited from use cases - and performance measures based on the actual  deployment}
  
  
\todo{this section contains preliminary content}

%Intro to evaluation
\todo{missing section introduction}
%Privacy and security as a second concern

\subsection{Analytic Validations}
The analytic part of the validation include formal or semi-formal validations and analysis of the different properties of the system. This validation tests will be interested, in particular, in comparing the implemented version of the Peer Manager with its ideal counterpart described during the first and second year deliverables (requirement compliance) and also finding out ``the cost of privacy'' by calculating the overhead in time and space that implementing these privacy measures brings to general systems.

The first results of this analytic validation activities will be reported during the third year review meeting of the project with the final results reported in D4.4 and the fourth year review meeting.

\subsubsection{Requirement compliance}
The requirement compliance of the peer manager comprises both evaluation of the general project requirements (found in the DoW and in D1.1??) and the compliance with the privacy requirements established in D4.1 and revised in D4.2.

For the compliance with the privacy requirements we plan to express a simplified version of the Peer Manager using the formalization of privacy sensitive systems found in [??]. We will then proceed to prove the extent to which the so represented Peer Manager is in fact compliant with the privacy requirements set in earlier version of this deliverable. 

The strategy for validation of the project requirements will include applied methods such as unit and integration tests of the Peer Manager implementation; along with performance tests of the integrate Peer Manager as part of the project's use cases. For the last case, performance evaluations in both time and space will also be performed and reported accordingly.  

\subsubsection{``Cost of privacy'' measurements}
This exercise will compare the time and space complexity of two Peer Managers, one without any privacy considerations and the other (as developed for the project) compliant with the set privacy requirements. Generalizing one step further, it is expected that this comparison would be able to show an estimate of the cost in time and space complexity for complying with these privacy requirements (that are now part or being discussed to become part of the EU privacy legislation) in a wide-range of relevant systems.

More in particular, the time/space complexity in both the privacy-less and the privacy-enabled Peer Managers will be compared independently for storing information and reading/accessing this information, 

\subsection{SmartShare Exploratory Survey}
SmartShare is a ridesharing application developed by the SmartSociety consortium as test and validation of several of the project's ideas.  A trial using the SmartShare application is planned during the second half of 2015 in multiple Italian municipalities.  

As part of the SmartShare trial, we plan to carry out an exploratory survey aimed initially to gauge the interest and knowledge of the participating users in privacy-related issues and technologies. Furthermore, based on these results we plan to adjust and create requirements related to the focused user activities and testing planned for year four.
 
The list of the questions being asked in this exploratory survey can be found in the Appendix~\ref{sec:smartshare-survey} and the results of this validation activity will be reported during the third year review of the project.

\subsection{Usage analysis of the SmartSociety Platform}
Through integration with the SmartSociety platform the peer manager will be used in several of the small-scale experiments organized by WP8 and also the Virtual Gamified Environment developed as part of WP9. We will take these opportunities to measure the real-use performance of the Peer Manager and, when possible, get information from the involved users.

\todo{Specific questions to be asked by WP4 towards the Virtual Gamified Environment to be added here}

The results of this validation activity will be part of the D4.4 and will be reported during the fourth year review of the project.

\subsection{Focused user activities and testing}
The final focused user activities relating to the peer manager and platform-wide privacy considerations will be carried out during the fourth year of the project. We plan to use all previous validations (specially the SmartShare Exploratory Survey) to inform and better adjust this validation activity. 

Based on this, further quantitative and qualitative user-related studies may be carried out, already aiming to validate not only the Peer Manager but also to draw project-wide and CAS-related conclusions of the exercise. 

The results of this validation activity will be part of the D4.4 and will be reported during the fourth year review of the project.
