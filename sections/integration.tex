The Peer Manager has been integrated in the SmartSociety platform
during the second year of the project, and released as component of
the version 1.0 of the platform. 

Currently the Peer Manager functionality
is used in both project-wide scenarios, i.e., SmartShare and
AskSmartSociety!~\cite{D8.2,D8.3}.

In general terms, the PM is mostly used as privacy-preserving
information service for peer-related information. In this sense, it is
at the center of a rather complex web of interactions with other
components. In particular, it is used by the Orchestration Manager to
identify suitable peers for carrying out a given task (composition
phase). It also provides information on how peers shall be contacted
for the negotiation process, the actual interaction being mediated by
the SmartCom middleware. The PM is tightly interacting with the
Context Manager, a component (developed within WP3) able to mine
streams of data to provide a near real-time representation of the
current user context. This situational information is integrated as
dynamic part of the user profile, and can be used for ensuring the
search process returns purposeful results. 

In this section we briefly describe how the PM is used in the context
of overall SmartSociety platform architecture, with reference to the
two aforementioned scenarios.

In both contexts, the PM is used to perform all operations related to
the management of collectives
(creation, retrieval of peers in a collective and retrieval of peer
information and contact method for communication
purposes~\cite{D8.2}). In SmartShare a collective is the set of peers
representing users taking part in a ride. In AskSmartSociety! a
collective is the set of peers selected as adequate (by the PM) to
answer a given question. These operations are typically requested by
the application peer (creation of collective) and by the SmartCom
communication middleware (retrieval of peers in a collective and peer information).

The PM search functionality plays a key role in the AskSmartSociety!
application scenario, where it is exploited to identify a suitable set
of peers who could best contribute to answer a given
question. This function is invoked by the OM and the response triggers
the execution of the negotiation process~\cite{D8.2}. In the context
of AskSmartSociety!, the ability of
the PM to handle transparently machine and human peers represent a key
feature, in that it enables full hybridity of the application. Also, the usage of the Concept Search approach, and in
particular the ability of the PM to perform semantic matching on
peers' attributes is instrumental in identifying human peers with the
right expertise to answer a given query, hence contributing
significantly to the quality of the answers collected.
